% this document is part of the DoCrimes project.
% Copyright 2022 the authors

\documentclass[modern]{aastex631}
\usepackage[utf8]{inputenc}
\usepackage[letterpaper]{geometry}

% typesetting issues
\sloppy\sloppypar\raggedbottom\frenchspacing

\begin{document}

\title{%
Testing cosmic isotropy and homogeneity
with \textsl{Gaia} quasars}
\author{David W. Hogg}
\affil{Center for Cosmology and Particle Physics, New York University}
\affil{Max-Planck-Institut f\"ur Astronomie}
\affil{Flatiron Institute, a division of the Simons Foundation}

\author{Kate Storey-Fisher}
\affil{Center for Cosmology and Particle Physics, New York University}

\date{2022 August}

\begin{abstract}\noindent
All cosmological models in general relativity are built on the assumption of large-scale isotropy and homogeneity.
These assumptions are testable.
Tests of cosmic isotropy and homogeneity are also strong tests of the quality of data or catalogs and their calibration.
Test precision increases as observations cover more solid angle, become more sensitive to distant objects, and are made with more uniform data.
The ESA \textsl{Gaia} Mission is primarily concerned with stars in the Milky Way, but it also observes millions of quasars, all sky (modulo Galactic extinction), to redshifts of around $4.5$.
It has produced the largest-ever quasar catalog by comoving volume, by far.
Here we use a carefully curated subsample of ZZZ \textsl{Gaia} quasars to show that the angular distribution, redshift distribution as a function of sky position, amplitude of clustering as a function of sky position, and counts of pairs as a function of separation (fractal dimension), are all consistent with large-scale isotropy and homogeneity at the precision possible given the size of the catalog and the amplitude of the large-scale structure.
Roughly speaking, the results can be summarized by saying that they show the Universe to be isotropic to better than XXX percent and homogeneous to better than YYY percent.
There is no sign of any North--South power asymmetry.
The isotropy results can be alternatively interpreted as a strong test of the spectrophotometric precision of the \textsl{Gaia} Catalog.
\end{abstract}

\keywords{%
calibration
---
catalogs
---
cosmic~isotropy
---
cosmological~principle
---
cosmology
---
large-scale~structure~of~the~Universe
---
quasars
}

\section{Introduction}

Hello World!

\section{Data}

\section{Method and results}

\section{Discussion}

\end{document}
